\section{Měření objemu stromu Ostatní stránky}
Následující seznam odkazuje na další stránky projektu:\begin{CompactList}
\item Webová aplikace projektu měření objemu řeziva.{\tt  [external]}
\item Manu�l k webov� aplikaci{\tt  [external]}
\item Zaříení na vozíku{\tt  [external]}
\item Manuál obsluhy{\tt  [external]}
\item \contentsline{section}{Komunikační protokol}{\pageref{comunication_protocol}}{}

\item \contentsline{section}{Práce s jednočipem}{\pageref{glob_prace_s_jednocipem}}{}

\item \contentsline{section}{Soubory}{\pageref{struktura}}{}

\end{CompactList}
